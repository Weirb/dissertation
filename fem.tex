\chapter{The finite element method}

This chapter discusses the well known and widely used method for solving partial differential equations, the finite element method (FEM).
To avoid building and adding another library to the already vast ecosystem of available FEM software, we use the multi-physics FEM library, \oomph.
From the website, \oomph is:
\begin{quote}
	an object-oriented, open-source finite-element library for the simulation of multi-physics problems. \cite{oomph}
\end{quote}
There are several high level features of any general finite element procedure that will be addressed below.
In addition to this, an overview of specific implementation details and features available in \oomph will also be covered.



% -----------------------------------------------------------

\section{Preliminaries}

We begin this chapter by discussing the necessary mathematical background for the finite element method.

Let us first define a problem in our context of interest.
A problem $(P)$ is one or more differential equations, together with appropriate boundary or initial conditions such that 

\iffalse WHAT DO WE NEED TO INCLUDE?

* Sobolev spaces
* Hilbert spaces 
* Banach spaces
* Norms/inner products

\fi



\subsection{Function spaces}
The test functions used in the upcoming weak formulation are members of a Sobolev space.
These function spaces are used to weight the residual of equation, 

The space of functions
\begin{align}
	H^1(\Omega) = \left\{ v | v \in L^2(\Omega), \, \nabla v \in (L^2(\Omega))^n \right\},
\end{align}
consists of those functions that are square-integrable and have a square-integrable derivative.



% -----------------------------------------------------------

\section{Nodes, elements, and meshes}

A mesh is a collection of elements; an element is a collection of nodes.

The building blocks of the FEM are surely the elements themselves, for which the method is named.
The elements are the 

Together with the elements, other fundamental pieces of the FEM are the nodes and the mesh.
Every element consists of multiple nodal points, each of which holds data required for solving the problem.
For example, nodes contain the global coordinate of their position within the mesh

A mesh is a collection of elements


There is one more detail of the hierarchy of entities for \oomph.
A \texttt{Problem} object contains details of a specific implementation for a problem, one of whose components is a mesh.
A \texttt{Node} objects contains multiple \texttt{Data} objects, holding the any or more of the data mentioned above.
This is simply an implementation detail, but an important one nonetheless. 


One of the fortunate things about the FEM is the discrete entities that are involved.
This lends itself to object-oriented programming (OOP) design principles particularly well.
Such a design strategy involves the definition of classes, and their instantiation as objects.
Furthermore, classes may be extended using inheritance, allowing for highly abstracted and generic classes, and on the other end, highly specific classes.
This works to both developers and users of the library advantage.
Developers are able to work independently of each other, knowing only how their respective objects should interact.
Users are able to extend previous work to suit their own needs.

Further discussion of OOP design principles is irrelevant and outside of the scope of this project.




% -----------------------------------------------------------

\section{Weak formulation}

The classical or strong formulation of a differential equation
The weak form of a differential equation is a relaxation of the conditions for the existence of a solution to the equation.

Given a test function $v$ from some function space $V$, we multiply our equation by $v$ and integrate over the domain.



It is also needed that the test function satisfies the 



Recall our PDE, the Fourier decomposed Helmholtz equation, \eqref{eqn:fhh}.
We will now derive the weak form of this equation as described above. Firstly by multiplying by a test function $v \in V$, then integrating over the domain $\Omega$.
For all $v \in V$,
\begin{align}
	-\int_\Omega \left(\nabla^2 u_N + \left(k^2-\frac{N^2}{r^2}\right)u_N \right) v = 0.
\end{align}
Using the product rule for the gradient operator $\nabla$, we rewrite the higher order derivatives as a sum of lower order terms.
\begin{align}
	-\int_\Omega \nabla \cdot \left( v \nabla u_N \right) - \nabla u_N \cdot \nabla v + \left(k^2-\frac{N^2}{r^2}\right)u_N v = 0.
\end{align}
Finally, the use of the of the divergence theorem yields
\begin{align}
	\int_\Omega \nabla u_N \cdot \nabla v - \left(k^2 - \frac{N^2}{r^2}\right) \int_\Omega u_N v - \int_{\partial\Omega} v \frac{\partial u_N}{\partial \vec{n}} = 0.
\end{align}
where $\vec{n}$ is the outward facing normal vector from the boundary $\partial \Omega$.

The boundary integral may be further split up into parts sorted by their respective boundary conditions.
If the portion of the boundary whose solution value is known is represented by $\partial \Omega_D$ and the remaining boundary is represented as $\partial \Omega_N$, then
\begin{align}
	\partial \Omega_D \cup \partial \Omega_N &= \partial \Omega. \\
	\partial \Omega_D \cap \partial \Omega_N &= \emptyset.
\end{align}
Given these exclusivity properties on the boundary, we may write the term for the boundary integral as
\begin{align}
	\int_{\partial\Omega} = \int_{\partial\Omega_D} + \int_{\partial\Omega_N}.
\end{align}
Since the test functions are exactly 0 on the Dirichlet portion of the boundary, it is the case that
\begin{align}
	\int_{\partial\Omega_D} v \frac{\partial u_N}{\partial \vec{n}} = 0.
\end{align}

The final step in the weak formulation of our equation results in
\begin{align}
	\int_\Omega \nabla u_N \cdot \nabla v - \left(k^2 - \frac{N^2}{r^2}\right) \int_\Omega u_N v - \int_{\partial\Omega_N} v \frac{\partial u_N}{\partial \vec{n}} = 0,
\end{align}
for all test functions $v\in V$.






% ----------------------------------------------------------

\subsection{Discretisation of boundary conditions}

\iffalse 
What are the boundary conditions that we have?
Sommerfeld, Neumann, PML..
How do we discretise these?
\fi







\section{Galerkin method}

In the Galerkin method, the test functions are the element basis functions.
This has the advantage that no 





\section{Rayleigh-Ritz method}

As an alternative to the Galerkin method presented above, we briefly present another formulation of the finite element method.
The Rayleigh-Ritz method begins with a functional, instead of the weak form.
Solving the differential equation $Lu=f$ is formally equivalent to minimising the functional
\begin{align}
	F(u) = \frac{1}{2}\left( Lu, u \right) - \left( f, u \right).
\end{align}

The two methods coincide when the linear operator $L$ is positive and self-adjoint and the function $F$ is quadratic.





\section{Putting it all together}

Once the equations have been 
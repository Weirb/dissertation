\chapter{Problem description}
\label{sec:problem}

\iffalse

Background definitions, notations, etc.
Problem definition
	- Relation to Helmholtz
Boundary condition
	- Sommerfeld
	- PML/ABC/DtN
Shifted Laplace operator


The Poisson problem:
	- Well behaved
	- Eigenvalues of the residual
	- Optimal method for finding solution

Helmholtz problem:
	- Misbehaved
	- Poor convergence
	- Indefinite discretisation matrix
\fi

We wish to solve the three dimensional Helmholtz equation in Fourier decomposed axisymmetric cylindrical coordinates.
Let us derive the discretisation by finite elements starting from the wave equation in Cartesian coordinates.

\section{Problem derivation}

Consider the three dimensional wave equation with wave speed $c$ in an unbounded domain,
\begin{equation}
	\frac{1}{c^2} \frac{\partial^2 u(x,y,z,t)}{\partial t^2} = \nabla^2 u(x,y,z,t). \label{eqn:wave}
\end{equation}
Assuming the existence of a separable solution where the time component of the solution is time-harmonic with frequency $\omega$,
we may write the real-valued solution as
\[
	u(x,y,z,t) = \Re \left( U(x,y,z)e^{-i\omega t} \right),
\]
where $U(x,y,z)$ is a complex-valued function \cite{oomph_hh}.
Substituting this into \eqref{eqn:wave}, we obtain the Helmholtz equation
\begin{equation}
	\nabla^2 U(x,y,z) + k^2 U(x,y,z) = 0, \label{eqn:hh}
\end{equation}
where $k=\omega/c$ is the wavenumber.

The mapping between Cartesian coordinates $(x,y,z)$ and cylindrical polar coordinates $(r,\varphi,z)$ is given by
\begin{align}
	x &= r\cos(\varphi), \\
	y &= r\sin(\varphi), \\
	z &= z.
\end{align}
The Jacobian of this mapping is given by
\begin{align}
	J = \left\vert \frac{\partial(x,y,z)}{\partial(r,\varphi,z)} \right\vert = r.
\end{align}

This mapping transforms our equation from $U(x,y,z)$ into $U(r,\varphi,z)$, 
which we may decompose into its Fourier components $u_N(r,z)$ around the $\varphi$ axis, so that
\begin{equation}
	U(r,\varphi,z) = \sum_{N=-\infty}^\infty u_N(r,z) e^{iN\varphi}. \label{eqn:full_eqn}
\end{equation}
By linearity of the Helmholtz equation, for each $N$, $u_N(r,z)$ must satisfy the Helmholtz equation.
This allows us to solve for each $N$ individually and then combine the solutions to form a full solution.

The Laplacian in cylindrical coordinates is given by
\begin{align}
	\nabla^2 = \frac{\partial^2 }{\partial r^2}
			 + \frac{1}{r} \frac{\partial }{\partial r}
			 + \frac{1}{r^2} \frac{\partial^2 }{\partial \varphi^2}
			 + \frac{\partial^2 }{\partial z^2}. \label{eqn:laplacian}
\end{align}
We also define the reduced Laplacian in cylindrical coordinates, consisting of only partial derivatives with respect to $r$ and $z$.
\begin{align}
	\nabla^2 = \frac{\partial^2 }{\partial r^2}
			 + \frac{1}{r} \frac{\partial }{\partial r}
			 + \frac{\partial^2 }{\partial z^2}. \label{eqn:laplacian_reduced}
\end{align}
It is not necessary to distinguish between the two operators \eqref{eqn:laplacian} and \eqref{eqn:laplacian_reduced}.
In the case of a function in $(r,\varphi,z)$, then the full cylindrical Laplacian is required.
Alternatively, for a function in $(r,z)$, the reduced cylindrical Laplacian is appropriate.
Therefore it should be implicit from the context of the equation.

Hence above, we wish to extract those terms consisting of $\varphi$ from \eqref{eqn:full_eqn}.
The partial derivatives with respect to $\varphi$ give
\begin{align}
	\frac{1}{r^2} \frac{\partial^2 }{\partial \varphi^2} \left( e^{i N \varphi}\right) = - \frac{N^2}{r^2} e^{i N \varphi}.
\end{align}
The full equation then becomes
\begin{align}
	\nabla^2 U(r,\varphi,z) 
		&= \nabla^2 \left[\sum_{N=-\infty}^\infty u_N(r,z) e^{i N \varphi}\right] \\
		&= \sum_{N=-\infty}^\infty \left[ 
										\nabla^2 u_N(r,z) - \frac{N^2}{r^2}u_N(r,z)
								   \right] e^{i N \varphi}.
\end{align}

Since we can solve each term in the series separately, we can extract a single equation for each $N$.
This gives us the Fourier decomposed Helmholtz equation,
\begin{equation}
	\nabla^2 u_N(r,z) + (k^2 - \frac{N^2}{r^2})u_N(r,z) = 0. \label{eqn:fhh}
\end{equation}
Now we have an equation in an axisymmetric domain, hence reducing the dimension of the problem.



\subsection{The Poisson problem}

In the case where $k=0$, the problem reduces to the Poisson problem.
Taking $k=0$ in equation \eqref{eqn:fhh}, we obtain
\begin{align}
	\nabla^2 U(r,\varphi,z) = \nabla^2 u_N(r,z) - \frac{N^2}{r^2} u_N(r,z) = 0. \label{eqn:fp}
\end{align}
This is the Fourier decomposed Poisson equation in cylindrical coordinates.

This equation is of interest because of the relation between the Helmholtz and Poisson equations.
Since we are interested in how multigrid performs when solving \eqref{eqn:fhh}, we will compare results with related problems.

Another related problem is the case when both $k=0$ and $N=0$.
This is the Poisson equation in reduced coordinates, with a solution depending only on $r$ and $z$.
To further aid in comparison between solutions, we will also consider this case.






% ----------------------------------------

\section{Complex-shifted Laplacian}










% ----------------------------------------

\section{Problem domain}

The full wave equation is defined in three spatial dimensions in cylindrical coordinates.
It is possible to solve the full equation

We consider the case where solutions are symmetric about the $z$ azis.
Hence we are able to decompose the solution into its Fourier components in the azimuthal direction.
By doing this, the dimension of the problem is reduced from three to two.
This simplification aids in the 

We introduce the Fourier wavenumber $N$ as an additional parameter to the system.


Upon reducing the equation

\begin{figure}[ht!]
	% Need an example image of a 3d cylinder - the domain for the problem -- away from the 
	% Also the 2d section of the cylinder
	\centering
	\subfloat[][Full cylindrical domain. \label{fig:full_domain}]{\includegraphics[draft]{images/placeholder}}\\
	\subfloat[][Reduced dimensional domain. \label{fig:reduced_domain}]{\includegraphics[draft]{images/placeholder}}
	\caption{Problem domain.\label{fig:problem_domain}}
\end{figure}

Figure \ref{fig:full_domain} represents an example of the domain in full 3D cylindrical coordinates.

Figure \ref{fig:reduced_domain} shows the dimension reduction from the Fourier decomposition.

The sharp edges at the corners of the domain are generally not a problem.
Absorbing boundary conditions take care of that.





% ----------------------------------------

\section{Boundary conditions}

To complete the problem formulation, boundary conditions must also be specified for the PDE.
Since our equation is of elliptic type, any appropriate Dirichlet, Neumann, or Robin conditions will suffice.
In addition to the boundary conditions, an additional condition on the solution must be imposed to ensure that the solution is unique since the domain of the problem is infinite.
The Sommerfeld radiation condition
\[
\lim_{|x|\rightarrow \infty} |x|^{\frac{n-1}{2}} \left( \frac{\partial}{\partial |x|} - ik \right) u(x) = 0,
\]
achieves this by ensuring that sources scatter to infinity and do not come from infinity \cite{sommerfeld}.




\subsection{Perfectly matched layers}

Perfectly matched layers (PML) are a method for handling infinite domains computationally, by damping or absorbing the wave-like solution in a layer outside of the region of interest.
Without such a tool, the truncated domain acts like a hard boundary, causing reflections of the solution back onto the domain.
These reflections are not physical, and so any solution found with these artifacts are useless for a physical model of waves.
The use of a PML will instead absorb the 


\begin{figure}[h]
	\centering
	% FIGURE FOR PML, show outside boundary 
	\includegraphics[draft]{images/placeholder}
	\caption{Perfectly matched layer \label{fig:pml}}
\end{figure}

% NEED TO MENTION AXIS-ALIGNED COORDINATES AND OTHERS

Inside of the PML, derivatives are mapped according to a complex coordinate transformation.
For our case in $r$ and $z$ coordinates, the derivatives get transformed to
\begin{align}
	\frac{\partial}{\partial r} \rightarrow \frac{1}{\gamma_r(r)}\frac{\partial}{\partial r}, \qquad \qquad
	\frac{\partial}{\partial z} \rightarrow \frac{1}{\gamma_z(z)}\frac{\partial}{\partial z},
\end{align}
where the damping factors $\gamma_r$ and $\gamma_z$ inside of the PML are given by
\begin{align}
	\gamma_r(r) &= 1 + \frac{i}{k} \frac{1}{\left| R_{\text{PML}} - r \right|}, \\
	\gamma_z(z) &= 1 + \frac{i}{k} \frac{1}{\left| Z_{\text{PML}} - z \right|},
\end{align}
where $R_{\text{PML}}$ and $Z_{\text{PML}}$ are the $r$ and $z$ coordinates of the PML 
Inside of the computational region, the PML is not active so the damping is not desirable.
Hence,
\begin{align}
	\gamma_r(r) = \gamma_z(z) = 1.
\end{align}



\begin{align}
	%
\end{align}

Given a computational domain 





% --------------------------------------

\section{An analytical solution}

In order to show that the computed solution of a numerical method is producing accurate results, a known solution is required to validate results.
The axisymmetric domain

Let us begin by assuming the existence of a separable solution,
\begin{align}
	u_N(r,z) = R(r)Z(z).
\end{align}
Substituting into \eqref{eqn:fhh}, we obtain
\begin{align}
	\left( \frac{\d^2 R(r)}{\d r^2} + \frac{1}{r} \frac{\d R(r)}{\d r} \right) Z(z) + \frac{\d^2 Z(z)}{\d z^2} R(r) + \left( k^2 - \frac{N^2}{r^2} \right) R(r) Z(z) = 0.
\end{align}
Dividing through by $R(r)Z(z)$, we can separate terms consisting of $r$ and $z$ on each side of the equation.
\begin{align}
	\left( \frac{\d^2 R(r)}{\d r^2} + \frac{1}{r} \frac{\d R(r)}{\d r} \right)\frac{1}{R(r)} + \left( k^2 - \frac{N^2}{r^2} \right) = - \frac{\d^2 Z(z)}{\d z^2} \frac{1}{Z(z)}.
\end{align}
Because both sides of this equation are equal expressions involving different variables, they must be equal to some constant, say $\lambda$, whose sign is yet to be determined.
Hence, we obtain two ordinary differential equations in $r$ and in $z$.
\begin{align}
	&\frac{\d^2 Z(z)}{\d z^2} + \lambda Z(z) = 0. \label{eqn:Z}\\
	&\frac{\d^2 R(r)}{\d r^2} + \frac{1}{r} \frac{\d R(r)}{\d r} + \left( k^2 + \lambda - \frac{N^2}{r^2} \right) R(r) = 0. \label{eqn:R}
\end{align}


First, we solve \eqref{eqn:R} for $R(r)$.
Consider the change of variables
\begin{align}
	\tilde{r} = r \sqrt{k^2 + \lambda}.
\end{align}
By the chain rule, derivatives in $\tilde{r}$ become
\begin{align}
	\frac{\d R}{\d r} &= \frac{\d \tilde{r}}{\d r} \frac{\d R}{\d \tilde{r}} = \sqrt{k^2 + \lambda} \frac{\d R}{\d \tilde{r}} \\
	\frac{\d^2 R}{\d r^2} &= \sqrt{k^2 + \lambda} \frac{\d}{\d \tilde{r}} \left[ \sqrt{k^2 + \lambda} \frac{\d R}{\d \tilde{r}} \right]
							= \left(k^2 + \lambda\right) \frac{\d^2 R}{\d \tilde{r}^2}
\end{align}
Substituting into \eqref{eqn:R}, 
\begin{align}
	\left(k^2 + \lambda\right) \frac{\d^2 R(\tilde{r})}{\d \tilde{r}^2} + 
	\frac{\left(k^2 + \lambda\right)}{\tilde{r}} \frac{\d R(\tilde{r})}{\d \tilde{r}} + 
	\left(k^2 + \lambda\right) \left( 1 - \frac{N^2}{\tilde{r}^2} \right) R(\tilde{r}) = 0.
\end{align}
Upon cancelling factors, we have the final equation, which may be recognised as Bessel's differential equation.
\begin{align}
	\frac{\d^2 R(\tilde{r})}{\d \tilde{r}^2} + 
	\frac{1}{\tilde{r}} \frac{\d R(\tilde{r})}{\d \tilde{r}} + 
	\left( 1 - \frac{N^2}{\tilde{r}^2} \right) R(\tilde{r}) = 0. \label{eqn:bessel_de}
\end{align}
The solution to \eqref{eqn:bessel_de} is a linear combination of Bessel functions of order $N$, and whose argument is $\tilde{r}=r \sqrt{k^2 + \lambda}$.
\begin{align}
	R(r) = c_1 J_N \left( r \sqrt{k^2 + \lambda} \right) + c_2 Y_N \left( r \sqrt{k^2 + \lambda} \right)
\end{align}


Now we turn to the solution of \eqref{eqn:Z}.
The form of the solution of $Z(z)$ is depending on the sign of $\lambda$.
Let $\mu$ be a positive real number.
If the sign is negative so that $\lambda=-\mu^2$, then $Z(z)$ is a linear combination of hyperbolic sine and cosine,
\begin{align}
	Z(z) = c_3 \sinh(\mu z) + c_4 \cosh(\mu z).
\end{align}
On the other hand, if the sign is positive, so that $\lambda=\mu^2$, then $Z(z)$ is a linear combination of sine and cosine,
\begin{align}
	Z(z) = c_3 \sin(\mu z) + c_4 \cos(\mu z).
\end{align}
Finally, if $\lambda=0$, then $Z(z)$ is the equation of a straight line,
\begin{align}
	Z(z) = c_3 z + c_4.
\end{align}

For the sake of simplicity, we can specify that $\lambda=0$.
We have then for the full general solution,
\begin{align}
	u_N(r,z) = R(r)Z(z) = \left( c_1 J_N \left( k r \right) + c_2 Y_N \left( k r \right) \right) \left( c_3 z + c_4 \right),
\end{align}
for any given constants $c_1,c_2,c_4,c_4$.

% We may simplify this expression further, by noting that up to a rescaling of constants, the sum of the Bessel functions can be written as Hankel functions

\chapter{Problem description}


\iffalse

Background definitions, notations, etc.
Problem definition
	- Relation to Helmholtz
Boundary condition
	- Sommerfeld
	- PML/ABC/DtN
Shifted Laplace operator


The Poisson problem:
	- Well behaved
	- Eigenvalues of the residual
	- Optimal method for finding solution

Helmholtz problem:
	- Misbehaved
	- Poor convergence
	- Indefinite discretisation matrix
\fi

We wish to solve the three dimensional Helmholtz equation in Fourier decomposed axisymmetric cylindrical coordinates.
Let us derive the discretisation by finite elements starting from the wave equation in Cartesian coordinates.

\section{Problem derivation}

Consider the three dimensional wave equation with wave speed $c$ in an unbounded domain,
\begin{equation}
	\frac{1}{c^2} \frac{\partial^2 u(x,y,z,t)}{\partial t^2} = \nabla^2 u(x,y,z,t). \label{eqn:wave}
\end{equation}
Assuming the existence of a separable solution where the time component of the solution is time-harmonic with frequency $\omega$,
we may write the real-valued solution as
\[
	u(x,y,z,t) = \Re \left( U(x,y,z)e^{-i\omega t} \right),
\]
where $U(x,y,z)$ is a complex-valued function \cite{oomph_hh}.
Substituting this into \eqref{eqn:wave}, we obtain the Helmholtz equation
\begin{equation}
	\nabla^2 U(x,y,z) + k^2 u(x,y,z) = 0, \label{eqn:hh}
\end{equation}
where $k=\omega/c$ is the wavenumber.

The mapping between Cartesian coordinates $(x,y,z)$ and cylindrical polar coordinates $(r,\varphi,z)$ is given by
\begin{align}
	x &= r\cos(\varphi), \\
	y &= r\sin(\varphi), \\
	z &= z.
\end{align}

This transforms our equation from $U(x,y,z)$ into $U(r,\varphi,z)$, 
which we may decompose into Fourier components $u_N(r,z)$ around the $\varphi$ axis, so that
\begin{equation}
	U(r,\varphi,z) = \sum_{N=-\infty}^\infty u_N(r,z) e^{iN\varphi}.
\end{equation}
By linearity of the Helmholtz equation, for each $N$, $u_N$ must satisfy the Helmholtz equation.

The Laplacian in cylindrical coordinates is given by
\[
	\nabla^2 = \frac{\partial^2 }{\partial r^2}
			 + \frac{1}{r} \frac{\partial }{\partial r}
			 + \frac{1}{r^2} \frac{\partial^2 }{\partial \varphi^2}
			 + \frac{\partial^2 }{\partial z^2}
\]

Splitting the Laplacian into terms not containing derivatives with respect to $\varphi$,
\begin{align}
	\nabla^2 U(r,\varphi,z) 
		&= \nabla^2 \left[\sum_{N=-\infty}^\infty u_N(r,z) e^{i N \varphi}\right] \\
		&= \sum_{N=-\infty}^\infty \left[ 
										\nabla^2 u_N(r,z) - \frac{N^2}{r^2}u_N(r,z)
								   \right] e^{i N \varphi}.
\end{align}

This gives us the Fourier decomposed Helmholtz equation,
\begin{equation}
	\nabla^2 u_N(r,z) + (k^2 - \frac{N^2}{r^2})u_N(r,z) = 0. \label{eqn:fhh}
\end{equation}





% ----------------------------------------

\section{Boundary conditions}

To complete the problem formulation, boundary conditions must also be specified for the PDE.
Since our equation is of elliptic type, any appropriate Dirichlet, Neumann, or Robin conditions will suffice.
In addition to the boundary conditions, an additional condition on the solution must be imposed to ensure that the solution is unique since the domain of the problem is infinite.
The Sommerfeld radiation condition
\[
\lim_{|x|\rightarrow \infty} |x|^{\frac{n-1}{2}} \left( \frac{\partial}{\partial |x|} - ik \right) u(x) = 0,
\]
achieves this by ensuring that sources scatter to infinity and do not come from infinity \cite{sommerfeld}.




\subsection{Perfectly matched layers}

Perfectly matched layers (PML) are a method for handling infinite domains computationally, by damping or absorbing the wave-like solution in a layer outside of the region of interest.
Without such a tool, the truncated domain acts like a hard boundary, causing reflections of the solution back onto the domain.
These reflections are not physical, and so any solution found with these artifacts are useless for a physical model of waves.
The use of a PML will instead absorb the 


\begin{figure}[h]
	\centering
	% FIGURE FOR PML, show outside boundary 
	\includegraphics[draft]{images/placeholder}
	\caption{Perfectly matched layer \label{fig:pml}}
\end{figure}

% NEED TO MENTION AXIS-ALIGNED COORDINATES AND OTHERS

Inside of the PML, derivatives are mapped according to a complex coordinate transformation.
For our case in $r$ and $z$ coordinates, the derivatives get transformed to
\begin{align}
	\frac{\partial}{\partial r} \rightarrow \frac{1}{\gamma_r(r)}\frac{\partial}{\partial r}, \qquad \qquad
	\frac{\partial}{\partial z} \rightarrow \frac{1}{\gamma_z(z)}\frac{\partial}{\partial z},
\end{align}
where the damping factors $\gamma_r$ and $\gamma_z$ inside of the PML are given by
\begin{align}
	\gamma_r(r) &= 1 + \frac{i}{k} \frac{1}{\left| R_{\text{PML}} - r \right|}, \\
	\gamma_z(z) &= 1 + \frac{i}{k} \frac{1}{\left| Z_{\text{PML}} - z \right|},
\end{align}
where $R_{\text{PML}}$ and $Z_{\text{PML}}$ are the $r$ and $z$ coordinates of the PML 
Inside of the computational region, the PML is not active so the damping is not desirable.
Hence,
\begin{align}
	\gamma_r(r) = \gamma_z(z) = 1.
\end{align}



\begin{align}
	%
\end{align}

Given a computational domain 
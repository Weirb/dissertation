\chapter{Results}

\def\figwidth{0.7\columnwidth}

\def \mgcol {red}
\def \lucol {black}

\def \helmholtzfolder {data/output_k_100_n_5}
\def \poissonfolder {data/output_k_0_n_5}
\def \axisymmpoisson {data/output_k_0_n_0}
\def \axisymhelmholtz {data/output_k_10_n_0}



The results of the numerical experiments carried out are detailed below.
We wish to compare the outcomes to existing literature of Cartesian Helmholtz as well as cylindrical Poisson problems.
The data collected were CPU timings of SuperLU and multigrid for the parameter pairs $(k^2,N)$ listed in table \ref{tab:params}.
Data were also gathered on the number of iterations taken for FGMRES to converge for the multigrid preconditioner.

Solutions were computed on a rectangular mesh of the problem domain, $\Omega = [1,2] \times [0,1]$.
The elements used were linear, quadrilateral elements, consisting of four nodes on each element.
The tolerance for FGMRES to accept the solution and break the iteration was set to be $1.0\times 10^{-10}$.

% BIG ISSUE -- HOW DO WE FIX THIS? MAYBE JUST SORT IT
% MAYBE MENTION LARGE N CAUSES ISSUES, HOW WE SOLVE IT, THEN WE JUST KEEP N SMALL FOR RESULTS
% Increasing $N$ causes issues for both multigrid and superlu.
% Multigrid behaves worse, but both still don't work.
% Can rectify this by shifting the computational domain away from the $z$-axis.

The machines used to run the numerical experiments had a total of 130GB of RAM.
Unfortunately this means that when the number of degrees of freedom of the problem reached $1.3\times 10^7$, equivalent to seven refinement levels, the computer ran out of memory.
Because of this, only six data points have beeen gathered for each individual problem, corresponding to the six levels of refinement that were able to run.
This appears to be sufficient to begin to see the trend in the data, as can be seen below.
The direct solver SuperLU is responsible for this drawback in the results, and multigrid performed adequately for even more levels of refinement.


\bgroup
\def\arraystretch{1.2}
\begin{table}[h]
    \centering
    \begin{tabular}{C{2cm}|C{2cm}}
        $k^2$ & $N$ \\\hline
        100 & 5 \\
        10 & 0 \\
        0 & 5 \\
        0 & 0
    \end{tabular}
    \caption{\label{tab:params} Parameter choices for timing study.}
\end{table}
\egroup



% ------------------------------------

\section{Solution}

% Picture of the solution
% Picture of the error (to show we solve the correct problem)
% Brief discussion of the two, don't need to go on


Blah blah the solution

We can see from figure \ref{fig:sol_error} that increasing the resolution of the mesh decreases the error.
Each additional level of initial refinement reduces the element width by a factor of two.
The specific gradient of the line.. % NEED TO FIX THIS.
As the error is decreasing upon refinement, we can assert that the computed solution approaches the exact solution.







% ------------------------------------

\section{Timings}

Multigrid is an optimal solver as the time complexity of the algorithm is linear in the number of degrees of freedom in the problem.
Direct solvers perform far worse in comparison, with scaling that is often at least quadratic in the degrees of freedom.
The particular direct solver we are comparing against is SuperLU, which is a sparse direct solver that takes advantage of the matrix structure.
In general then, we expect multigrid to outperform SuperLU.

Figures \ref{fig:time1} and \ref{fig:time2} compare the run times of multigrid and SuperLU for the $N=5$ and $N=0$ cases for the Poisson and Helmholtz equations.
It is clear from these figures that multigrid eventually outperforms SuperLU, despite the lacking number of data points.
However, there is a significant setup cost in using multigrid to solve these problems.
This 

For the $N=0$ problems in figure \ref{fig:time2}, the point at which multigrid begins to outperform SuperLU comes much sooner.
This point is at approximately $8.0\times 10^4$ degrees of freedom for both Poisson and Helmholtz problems.

The gradient of the line through the multigrid data points on the log-log axis is approximately equal to one for all of the problems.
This is precisely in line with the theoretical result of the time-complexity.
For the gradient of the line through the SuperLU data, the value is approximately two.

When choosing which solver between the two to use, the primary consideration should be the size of the problem.
For a sufficiently large number of degrees of freedom, multigrid is a far superior choice.
SuperLU will require more memory and take more time to finish execution than multigrid for every problem tested here.








% ------------------------------------

\section{Iterations}

The number of FGMRES iterations to reach convergence tolerance should remain relatively constant upon refinement of the mesh.
This was demonstrated by Elman, Ernst, and O'Leary in \cite{elman} for the Helmholtz problem in Cartesian coordinates.
Table \ref{tab:iter} shows the iteration count for several different refinement levels.
It can be seen here that our results match closely with the literature, where we obtain relatively similar iteration counts for the solving problem.

In addition, we provide iteration counts for the Poisson problem in table \ref{tab:poissoniter}.
We see that the behaviour is similar to the multigrid case.


\bgroup
\def\arraystretch{1.2}

\begin{table}[ht!]
    \centering
    \csvstyle{myTableStyle}{
        longtable=c | c c c c c, % Specify column formatting
        table head={ % Table header is the first line
            \multirow{2}{*}{$N_\mathrm{dofs}$} & \multicolumn{5}{c}{$k^2$} \\
            & 0 & 10 & 20 & 50 & 100 \\ \hline
        },
        table foot={ % Table footer is the last line
            \caption{\label{tab:iter1} Number of iterations taken for FGMRES to converge to tolerance for $N=5$.}
        },
        late after line=\\,
        no head, % Does the data file have a header?
        separator=comma
    }
    \csvreader[myTableStyle]{data/iteration_data.dat}
        {1=\ci,2=\cii,3=\ciii,4=\civ,5=\cv,6=\cvi}
        {\ci & \cii & \ciii & \civ & \cv & \cvi}
\end{table}

\begin{table}[ht!]
    \centering
    \csvstyle{myTableStyle}{
        longtable=c | c c c c c, % Specify column formatting
        table head={ % Table header is the first line
            \multirow{2}{*}{$N_\mathrm{dofs}$} & \multicolumn{5}{c}{$k^2$} \\
            & 0 & 10 & 20 & 50 & 100 \\ \hline
        },
        table foot={ % Table footer is the last line
            \caption{\label{tab:iter1} Number of iterations taken for FGMRES to converge to tolerance for $N=0$.}
        },
        late after line=\\,
        no head, % Does the data file have a header?
        separator=comma
    }
    \csvreader[myTableStyle]{data/iteration_data.dat}
        {1=\ci,2=\cii,3=\ciii,4=\civ,5=\cv,6=\cvi}
        {\ci & \cii & \ciii & \civ & \cv & \cvi}
\end{table}

\egroup








% ----------------------------
% SOLUTION PLOTS
% ----------------------------

\begin{figure}

    \centering
    
    % Plot of the solution
    \subfloat[][Plot of the solution of the Helmholtz equation for $k^2=$ and $N=$.]{
        \includegraphics[width=\figwidth]{images/placeholder}
    }

    % Plot of the error
    \subfloat[][Behaviour of the error between the exact and computed solution by multigrid for the Helmholtz problem with $k^2=100,N=5$ for varying levels of initial refinement.]{
        \label{fig:sol_error}
        
        \begin{tikzpicture}
        \begin{loglogaxis}[
        xlabel={$h$},
        ylabel={Error},
        xmin=0.01, xmax=1.1,
        grid=major,
        width=\figwidth
        ]
        
        % Plot the data
        % \addplot [color=red, mark=x] table {\helmholtzfolder/error.dat};

        Calculate gradient, draw the triangle
        \addplot[color=\mgcol, mark=x] table[y={create col/linear regression={y=1}}] {\helmholtzfolder/error.dat} 
        coordinate [pos=0.8] (A)
        coordinate [pos=0.3] (B);
        \xdef\slope{\pgfplotstableregressiona} % save the slope parameter
        \draw (B) -| (A)  % draw the opposite and adjacent sides of the triangle
        node [pos=0.25, anchor=south] {1} % label the horizontal line
        node [pos=0.75,anchor=east] {\pgfmathprintnumber{\slope}} %label the vertical line
        ;

        \end{loglogaxis}
        \end{tikzpicture}
    }
    \caption{}
\end{figure}


% ----------------------------
% TIMING PLOTS
% ----------------------------

\begin{figure}
    
    % Full equations
    % time vs #dofs

    \centering
    
    \subfloat[][Helmholtz equation with $k^2=100$.]{
        \begin{tikzpicture}
        \centering
        \begin{loglogaxis}[
        xlabel={Degrees of freedom},
        ylabel={Time taken to solve (s)},
        grid=major,
        width=\figwidth,
        legend style={at={(0.02,0.98)}, anchor=north west}
        ]
        
        % Plot mg
        \addplot [color=\mgcol, mark=x] table [x index=1, y index=0] {\helmholtzfolder/mg_time.dat};
        \addlegendentry{Multigrid}
        
        % Plot superlu
        \addplot [color=\lucol, mark=x] table [x index=1, y index=0] {\helmholtzfolder/superlu_time.dat};
        \addlegendentry{SuperLU}

        \end{loglogaxis}
        \end{tikzpicture}
    }


    \subfloat[][Poisson equation.]{
        \centering
        \begin{tikzpicture}
        \centering
        \begin{loglogaxis}[
        xlabel={Degrees of freedom},
        ylabel={Time taken to solve (s)},
        grid=major,
        width=\figwidth,
        legend style={at={(0.02,0.98)}, anchor=north west}
        ]
        
        % Plot mg
        \addplot [color=\mgcol, mark=x] table [x index=1, y index=0] {\poissonfolder/mg_time.dat};
        \addlegendentry{Multigrid}
    
        % Plot superlu
        \addplot [color=\lucol, mark=x] table [x index=1, y index=0] {\poissonfolder/superlu_time.dat};
        \addlegendentry{SuperLU}
    
        \end{loglogaxis}
        \end{tikzpicture}
    }
    \caption{\label{fig:time1} Comparison of the time taken to solve the Helmholtz and Poisson equations using the multigrid and SuperLU solvers for $N=5$.}
\end{figure}



\begin{figure}[h]
    
    % Axisymmetric equations
    % time vs #dofs

    \centering
    
    \subfloat[][Helmholtz equation with $k^2=10$.]{
        \centering
        \begin{tikzpicture}
        \begin{loglogaxis}[
        xlabel={Degrees of freedom},
        ylabel={Time taken to solve (s)},
        grid=major,
        width=\figwidth,
        legend style={at={(0.02,0.98)}, anchor=north west}
        ]
        
        % Plot mg
        \addplot [color=\mgcol, mark=x] table [x index=1, y index=0] {\axisymhelmholtz/mg_time.dat};
        \addlegendentry{Multigrid}
    
        % Plot superlu
        \addplot [color=\lucol, mark=x] table [x index=1, y index=0] {\axisymhelmholtz/superlu_time.dat};
        \addlegendentry{SuperLU}
    
        \end{loglogaxis}
        \end{tikzpicture}
    }

    \subfloat[][Poisson equation ($k^2=0$).]{
        \centering
        \begin{tikzpicture}
        \centering
        \begin{loglogaxis}[
        xlabel={Degrees of freedom},
        ylabel={Time taken to solve (s)},
        grid=major,
        width=\figwidth,
        legend style={at={(0.02,0.98)}, anchor=north west}
        ]
        

        \addplot [color=\mgcol, mark=x] table [x index=1, y index=0] {\axisymmpoisson/mg_time.dat};
        \addlegendentry{Multigrid}
    
        \addplot [color=\lucol, mark=x] table [x index=1, y index=0] {\axisymmpoisson/superlu_time.dat};
        \addlegendentry{SuperLU}
    
        \end{loglogaxis}
        \end{tikzpicture}
    }

    \caption{\label{fig:time2} Time taken to solve the axisymmetric versions of the Poisson and Helmholtz equations with $N=0$.}
\end{figure}
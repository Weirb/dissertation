\chapter{Results}

\def\figwidth{0.7\columnwidth}


What results did we gather?
What machine did we use?

Need a general discussion for each

Increasing $N$ causes issues for both multigrid and superlu.
Multigrid behaves worse, but both still don't work.
Can rectify this by shifting the computational domain away from the $z$-axis.


Solutions below were computed on a rectangular mesh of the problem domain, $\Omega = [1,2]\times [0,1]$.
The elements used were linear, quadrilateral elements, consisting of four nodes each.

The machine running the results



% ------------------------------------

\section{Solution}

% Picture of the solution
% Picture of the error (to show we solve the correct problem)
% Brief discussion of the two, don't need to go on


Blah blah the solution

We can see from figure \ref{fig:sol_error} that increasing the resolution of the mesh decreases the error.
Each additional level of initial refinement reduces the element width by a factor of two.
The specific gradient of the line..


\begin{figure}
    \centering

    \begin{tikzpicture}
        \centering
    \begin{loglogaxis}[
    xlabel={$h$},
    ylabel={Error},
    xmin=0.01, xmax=1.1,
    grid=major,
    width=\figwidth
    ]
    
    % Plot the data
    \addplot [only marks,fill=black] table {data/helmholtz_error.dat};

    % Calculate gradient and draw the triangle
    \addplot table[y={create col/linear regression={y=1}}] {data/helmholtz_error.dat} 
    coordinate [pos=0.8] (A)
    coordinate [pos=0.3] (B);
    \xdef\slope{\pgfplotstableregressiona} % save the slope parameter
    \draw (B) -| (A)  % draw the opposite and adjacent sides of the triangle
    node [pos=0.25, anchor=south] {1} % label the horizontal line
    node [pos=0.75,anchor=east] {\pgfmathprintnumber{\slope}} %label the vertical line
    ;

    \end{loglogaxis}
    \end{tikzpicture}
    \caption{\label{fig:sol_error} Behaviour of the error between the exact and computed solution by multigrid for the Helmholtz problem with $k^2=100,N=5$ for varying levels of initial refinement.}
\end{figure}




% ------------------------------------

\section{Timings}




\begin{figure}
    
    % Helmholtz equation 
    % time vs #dofs

    \centering
    \begin{tikzpicture}
    \centering
    \begin{loglogaxis}[
    xlabel={Degrees of freedom},
    ylabel={Time taken to solve (s)},
    grid=major,
    width=\figwidth,
	legend style={at={(0.02,0.98)}, anchor=north west}
    ]
    
    \addplot [color=black, mark=x] table [x index=1, y index=0] {data/helmholtz_mg_time.dat};
    \addlegendentry{Multigrid}

    \addplot [color=red, mark=x] table [x index=1, y index=0] {data/helmholtz_superlu_time.dat};
    \addlegendentry{SuperLU}

    \end{loglogaxis}
    \end{tikzpicture}
    \caption{Time taken to solve the Helmholtz equation with $k^2=100$ and $N=5$.}
\end{figure}





% ------------------------------------

\section{Iterations}

\begin{figure}
    
    % Helmholtz equation 
    % Iteration vs #dofs

    \centering
    \begin{tikzpicture}
    \centering
    \begin{axis}[
    xlabel={Degrees of freedom},
    ylabel={Number of iterations},
    xmode=log,
    grid=major,
    width=\figwidth,
	legend style={at={(0.02,0.98)}, anchor=north west}
    ]

    \addplot [color=red, mark=x] table [x index=1, y index=0] {data/helmholtz_mg_iter.dat};

    \end{axis}
    \end{tikzpicture}
    \caption{Number of FGMRES iterations against degrees of freedom, using multigrid as a preconditioner.}
\end{figure}


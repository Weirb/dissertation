\chapter{Introduction}

% Applications and who cares
The fundamental nature of waves means that waves appear in many different contexts in science, engineering, and industry.
For example, in electromagnetics, Maxwell's equations describe the physical properties of waves generated from electric and magnetic fields.
In acoustics, the propagation of sound waves may be modelled by the same equations.
Modelling of seismic waves can aid in the search for materials or objects underground and help to better understand earthquakes.
Understanding and modelling the scattering of waves is of great importance and practical use.
The equation describing the behaviour of waves in all of these applications is the wave equation, given by
\begin{align}
	\nabla^2 u = \frac{1}{c^2}\frac{\partial^2 u}{\partial t^2}, \label{eqn:first}
\end{align}
with wave speed $c$.
This is a time-dependent, second order, hyperbolic partial differential equation.
% the wave equation has widespread application and appears in
% The specific physics equation
The governing equation and the problem of interest in this project is the Helmholtz equation.
\begin{align}
	\nabla^2 u + k^2 u = 0.
\end{align}
The Helmholtz equation is a second order, elliptic type partial differential equation that arises when seeking separable solutions of the wave equation.
The physical interpretation of the solution $u$ depends on the particular application, but in general represents the amplitude of the wave with wavenumber $k$.
For our application, $u$ represents the acoustic pressure at a point in the domain.

Determining properties of an object located in a fluid is a problem applicable to many industries and fields.
For example, (Thales, defence apps)

% MAYBE CHANGE THIS, NON INFINITE CYLINDER
Consider a cylindrical object located within an infinite fluid.
We can represent the whole domain using cylindrical coordinates $(r, \phi, z)$, where $0\leq r < \infty$, $0 \leq \phi < 2\pi$, and $-\infty < z < \infty$.
If the centre of the object is located at the origin with some , then the domain contains a cylindrical hole at the origin.
Assuming symmetry of the domain about the vertical axis
By decomposing the solution into Fourier components in the azimuthal direction, we reduce the full cylindrical domain into 


% Boundary conditions
We must also specify boundary conditions for the surface of the object, as well as any other boundaries in the domain.
For an elliptic problem, all boundary edges must be specified with either Dirichlet, Neumann, or Robin conditions.
A Dirichlet condition, where the known value of the solution is given on the boundary, $u|_{\partial\Omega}=u_0$.
This type of boundary is known as soft, and can be interpreted as the boundary of an elastic wall that is displaced in response to a force on the outside.
A value of zero on the boundary is known as a pressure release boundary.
If the value here is nonzero, the boundary experiences some force from the other side.
For a Neumann boundary, the normal derivative $\vec{n}\cdot \nabla u |_{\partial \Omega} = \partial u / \partial \vec{n} |_{\partial \Omega}$ is given on the boundary.
The physical interpretation of this is a hard boundary, that does not move in response to a force.
Lastly a Robin condition, giving both the value of the solution and the normal derivative on the boundary can be interpreted as a combination of the two above.



% Sommerfeld + truncation
In addition to boundary conditions, an additional condition on the solution must be imposed to ensure that the solution is unique since the domain of the problem is infinite.
The Sommerfeld radiation condition in three dimensions is given by
\begin{align}
	\lim_{|x|\rightarrow \infty} |x| \left( \frac{\partial}{\partial |x|} - ik \right) u(x) = 0. \label{eqn:sommerfeld}
\end{align}
This achieves this requirement by ensuring that sources scatter to infinity and do not come from infinity \cite{sommerfeld}.

Because the problem is defined on an infinite domain, we must perform a truncation of the domain to find the solution numerically.
Choosing some appropriately sized region of interest surrounding the object, we restrict the domain to this region.
However, this causes problems with waves interacting with the boundary of the truncated domain.
Spurious waves from reflections on the outer walls propagate through the domain and the resulting solution is no longer correct.
It is possible to deal with these reflections by introducing an artifical absorbing layer to the boundary of the domain.
Several such layers exist, for instane absorbing boundary conditions (ABCs); Dirichlet-to-Neumann condtions (DtNs); and perfectly matched layers (PMLs).
These all attempt to dampen waves in the artificial layer surrounding the domain so that there are no reflections back into the domain.



% ----------------------------------------------

% How do we solve the problem
\iffalse
	Discretise
	Direct vs Iterative solver (Krylov subspace)
	Multigrid
	Preconditioners
	CSLP
	MG as a preconditioner
\fi

% FEM Discretisation
In order to solve the problem, we will use the C++ library \oomph to find the solution using the finite element method.
This process will involve the discretisation of the equations on each element of the domain with the resulting system being assembled.
The result of the finite element discretisation is an $n\times n$ linear system of equations,
\[
	A x = b,
\]
which requires a solver in order to compute the solution.
In \oomph, the default linear solver is SuperLU, however this is also a direct solver.
This has the benefit that the computed solution to the above equation is exact, modulo any floating point errors.
But the computational cost is prohibitively expensive as the size of the problem increases.

% Direct vs Iterative methods
For small $A$, it is possible to use direct methods to solve these exactly (in exact arithmetic).
As is often the case in practice, the matrices from the finite element discretisation are too large to compute by direct methods.
Instead, many iterative methods have been developed to approximate the solution instead.
A particular class of methods, Krylov subspace methods, form the solution by building iterates from elements in a space  of successive powers of the matrix $A$ \cite{leveque}.
The Krylov space of dimension $k$ is 
\[
	\mathcal{K}_k(A, r_0) = \mathrm{span}(r_0, Ar_0, A^2 r_0, \ldots, A^{k-1} r_0).
\]
Such methods include the conjugate gradient and GMRES methods.

% Preconditioning
This equation is generated from the finite element discretisation of the problem.
Linear systems of equations are everywhere, so there are a choice of several different methods to use to solve.
Direct methods are too costly to use since in general, the dimension of the problem is large.
Iterative methods, such as Krylov subspace methods, are a good alternative to use.
However the performance of such methods is poor without a preconditioner.
Hence, we would like a preconditioner that improves the performance of the iterative method.

Applying a preconditioning matrix $P$ on the left, we have
\begin{align}
	P^{-1} A x = P^{-1} b, \label{eqn:precond_left}
\end{align}
which solves the above system exactly for the case $P=A^{-1}$. 
This is optimal as it solves the original problem, however it is costly to obtain.
Instead, we would like a alternative such that $P$ is cheap to find and the preconditioned system \eqref{eqn:precond_left} is easier to solve than the original problem.

Alternatively, a preconditioning matrix can be applied on the right.
In this case, we have
\begin{align}
	A P^{-1} P x = b. \label{eqn:precond_right}
\end{align}
This can be solved in two steps, 
\begin{align}
	P x = y, \\A P^{-1} y = b.
\end{align}


Many general preconditioning techniques have been developed that work for a wide variety of problems.
For example, preconditioning matrices based on the splitting of the matrix $A$ are simple to use and cheap to compute.
More specific preconditioning techniques have been developed in the case of certain problems.
These tend to be more complicated to use, but often will come with better performance.
As a first attempt, it is a good idea to use a simple method to see the impact on the convergence times.
Only in the case where optimisation is crucial to the problem or a general preconditioner does not work, a specific preconditioner should be developed.


% Multigrid
% Need more at the beignning
% MULTIGRID http://www.mgnet.org/mgnet/books/Wesseling/bookr-orig.pdf
Wesseling describes the historical development of multigrid methods from their inception in 1967 up to 1985 \cite{wesseling}.
It is stated that the first mention of multigrid was by the Russian mathematician Fedorenko.
In his paper, Fedorenko developed a method for solving elliptic type partial differential equations, which he called the ``the method of alternating direction'' \cite{fedorenko}.
Later, Hackbusch independently developed multigrid methods and presented the rigorous mathematical framework for multigrid.

Since multigrid was developed in order to find solutions to elliptic problems, the majority of work has been focused in that area.
As the prototypical example of an elliptic PDE is Poisson's equation, we can be certain that the mathematical framework for solving this is set.
The difficulty lies with the Helmholtz equation, which can be thought of a perturbation by $k^2$ of the Laplace operator in the Poisson equation.
As both the Poisson and Helmholtz equations are both of elliptic type, we should expect behaviour to be similar.
However, this is not the case.

The performance of multigrid methods is excellent for the related Poisson problem,
\begin{equation}
	\nabla^2 u = f. \label{eqn:poisson}
\end{equation}
Indeed, it is an optimal solver with time-complexity of $O(n)$ for a problem of size $n\times n$.
For sufficiently small values of $k$, the Helmholtz equation is well approximated by Poisson's equation.
In addition, the matrix arising from the finite element discretisation is well-conditioned.
As $k$ becomes larger, however, the matrix becomes indefinite and more and more ill-conditioned.
This causes multigrid to fail to converge to a solution.
To handle the ill-conditioned system, a preconditioner can be used so that the problem may be more easily solved by numerical methods.

The multigrid solver for the Helmholtz equation exists for the equation in Cartesian coordinates in \oomph.
We would like to explore the performance of the solver for the Fourier decomposed version of the Helmholtz equation in axisymmetric cylindrical coordinates.
Following this, we will also examine the performance of the solver using PMLs to handle reflections on the discretised boundary.


% Multigrid as a preconditioner
Multigrid methods may also be used as a preconditioner for other iterative methods.
With a reduced

Since at each iteration of the outer iterative method the preconditioner will change, the iterative method must be `flexible'.
Saad created the Flexible GMRES iterative method, FGMRES \cite{fgmres}.
This is the solver used to solve the linear system 


% CSLP
Work on overcoming the issue for larger values of $k^2$ has been successful.
The current preferred method for solving the problem is to precondition a Krylov subspace method using a complex-shifted laplacian preconditioner (CSLP).
Convergence results obtained by authors Erlangga, Oosterlee, and Vuik show that among the class of preconditioners discussed, the CSLP is the most effective in preconditioning the Helmholtz problem for large wavenumber \cite{cslp1}.
They showed... % DISCUSSION OF THE FIRST PAPER

In a second paper by the same authors, they go on to discuss the CSLP in more detail, and in particular, applied to multigrid methods \cite{cslp2}.
% DISCUSSION OF THE SECOND PAPER









% -----------------------------------------------

\section{Project outline}

Chapter one contains the introduction to the project.
This has hopefully given you, the reader, the modern and historical context of the problem of interest, as well as the aims and objectives to motivate the rest of this dissertation.

The focus of chapter two is to derive the mathematics of the problem.
Here we will give the necessary mathematical overview and description of the problem and the derivation of equations.
Also here will be issues associated with the practicalities of finding a solution.

Chapter three aims to cover the necessary high level facets of the finite element method for the the problem in question.
Also covered are relevant references to the FEM implementation within \oomph and how the mathematical description of the FEM is transferred to an object-oriented code able to solve PDEs numerically.

Chapter four will motivate multigrid and outline the mathematics of the multigrid algorithm for solving the Helmholtz equation.
In particular, we discuss the application of multigrid in finding the solution to the finite element discretisation of the Helmholtz equation.

The project will come together at the end with a discussion of the results.
Here, a summary of findings and an overview of the work completed in the project will be given, as well as comparison to related problems and the current literature.

Finally, we will complete the dissertation with concluding remarks and outline steps for moving forward from the end of the project.
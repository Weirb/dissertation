\iffalse Brief outline of the chapter below

Background:
	What is HH equation?
	What does it model?
	How is it derived?
	Boundary conditions
	- 	Sommerfield radiation condition
	
Finite element method:
	Discretisation of HH
	Discretisation of infinite domains
		D,N,radiation
	ABC, DtN, PML

Linear solvers:
	Iterative and direct
	Krylov methods
	Preconditioners
	How does this relate to Poisson and HH?
	- 	Small wavenumbers good convergence

Project direction:
	Where are we going? What to do...?
	Multigrid in non-cartesian coordinates
	Fourier decompostion of HH
	Explore implementation of PMLs
\fi


\section{Background}

\iffalse
Motivation.
Who cares? Why is it difficult?
What are the benefits?
Need some citations? Names in the field...
\fi



Because of the fundamental nature of waves, the wave equation has widespread application and appears in several contexts in science, engineering, and industry.
For example, in electromagnetics, Maxwell's equations describe the physical properties of waves generated from electric and magnetic fields.
In acoustics, the propagation of sound waves may be modelled by the same equations.
Also, modelling of seismic waves can help to better understand earthquakes and aid in the search for materials or objects underground.
Understanding and modelling scattering of waves is of great importance and practical use.

<<CITATION>>
Work in the field of waves began as early as with d'Alembert and Euler in the middle of the 18$^\text{th}$ century.
The motivation behind these ideas was the development of optics and the propagation of light.
This in turn led to the invention of telescopes and microscopes and paved the road for further scientific discovery.

The work in this project will primarily focus on the related partial differential equation, the Helmholtz equation.
The wave and Helmholtz equations are closely linked, and study of one can lead to insight in the other.
When seeking separable solutions of the wave equation, the Helmholtz equation appears as a time-independent form of the wave equation.
The solutions of the time-independent equation represent steady state solutions.

The specific problem 


\section{Finite element method}

In order to solve the problem, we will use the C++ library \texttt{oomph-lib} to find the solution using the finite element method.
This process will involve the discretisation of the equations on each element of the domain with the resulting system being assembled.

Because the problem is defined on an infinite domain, we must truncate the domain to find the solution numerically.
This causes a problem with the waves interacting with the boundary of the domain, which produces spurious reflections and the resulting solution to become meaningless.
It is possible to deal with these reflections by introducing an artifical absorbing layer to the boundary of the domain.
Several such layers exist, including absorbing boundary conditions (ABCs); Dirichlet-to-Neumann condtions (DtNs); and perfectly matched layers (PMLs).

The final result of the finite element method is a linear system, which requires a solver in order to compute the solution.

\section{Iterative linear solvers}

Consider the linear system
\[
	A u = f.
\]
For small $A$, it is possible to use direct methods to solve these exactly (in exact arithmetic).
As is often the case in practice, the matrices from the finite element discretisation are too large to compute by direct methods.
Instead, many iterative methods have been developed to approximate the solution instead.
A particular class of methods, Krylov subspace methods, form the solution by building iterates from elements in a space composed of successive powers of the matrix $A$ \cite{leveque}.
The Krylov space of dimension $k$ is 
\[
	\mathcal{K}_k(A, r_0) = \mathrm{span}(r_0, Ar_0, A^2 r_0, \ldots, A^{k-1} r_0).
\]
For example, the conjugate gradient method forms an orthogonal basis consisting of search directions.
Unfortunately, a restriction on the convergence of the CG algorithm is that the matrix $A$ must be symmetric and positive definite.
This is often not the case for a general problem.

The performance of multigrid methods is excellent for the related Poisson problem.
For small values of $k$ in \eqref{eqn:hh}, the matrix arising from the finite element discretisation is well-conditioned.
As $k$ becomes larger, however, the matrix becomes indefinite and multigrid fails to converge to a solution.

To handle the ill-conditioned system, a preconditioner can be used so that the problem may be more easily solved by numerical methods.
Since multigrid is an effective solver for the Poisson problem, it is instead conceivable that few multigrid iterations could be used as a preconditioner for the Helmholtz problem.







\section{Project outline}

\iffalse
What is each chapter doing? What do we hope to accomplish? 
\fi

The motivation for this thesis is 

The multigrid solver for the Helmholtz equation exists for the equation in Cartesian coordinates in \texttt{oomph-lib}.
We would like to explore the performance of the solver for the Fourier decomposed version of the Helmholtz equation in axisymmetric cylindrical coordinates.
Following this, we will also examine the performance of the solver using PMLs to handle reflections on the discretised boundary.

The numerical solution of the Helmholtz equation 
% \chapter{Introduction}

\iffalse Brief outline of the chapter below

Background:
	What is HH equation?
	What does it model?
	How is it derived?
	Boundary conditions
	- 	Sommerfield radiation condition
	
Finite element method:
	Discretisation of HH
	Discretisation of infinite domains
		D,N,radiation
	ABC, DtN, PML

Linear solvers:
	Iterative and direct
	Krylov methods
	Preconditioners
	How does this relate to Poisson and HH?
	- 	Small wavenumbers good convergence

Project direction:
	Where are we going? What to do...?
	Multigrid in non-cartesian coordinates
	Fourier decompostion of HH
	Explore implementation of PMLs
\fi


\section{The Helmholtz equation}

Consider the wave equation in $n=2,3$ dimensions, in an infinite 
domain,
\[
	\frac{1}{c^2} \frac{\partial^2 u}{\partial t^2} = 
		\nabla^2 u,\addtag \label{eqn:wave}
\]
where $\nabla^2$ is the Laplacian operator and $c$ is the wavespeed.
If we assume that the real solution $u(\mathbf{x},t)$ is time-harmonic, that is, the time-dependence of the solution is sinusoidal with a single frequency $\omega$, then we can write
\[
	u(\mathbf{x},t) = 
		\Re \left( u(\mathbf{x})e^{-i\omega t} \right),
\]
where $u(\mathbf{x})$ is a complex-valued function \citep{oomph_hh}.
Substituting this into \eqref{eqn:wave}, we obtain the Helmholtz equation
\[
	\nabla^2 u(\mathbf{x}) + k^2 u(\mathbf{x}) = 0, \addtag \label{eqn:hh}
\]
where $k=\omega/c$ is the wavenumber.

For the problem to be well-posed, boundary conditions must also be specified for the equation.
Any appropriate Dirichlet, Neumann, or Robin conditions will 
In addition to the boundary conditions, an additional condition on the solution must be imposed to ensure that the solution is unique. 
The Sommerfeld radiation condition \citep{sommerfeld}
\[
\lim_{|x|\rightarrow \infty} |x|^{\frac{n-1}{2}} \left( \frac{\partial}{\partial |x|} - ik \right) u(x) = 0.
\]





\section{Finite element method}

\begin{enumerate}
\item Linear system from discretisation of 
\item Discretise infinite domain 
\end{enumerate}


\section{Iterative linear solvers}

Linear systems arise from the discretisation of finite element equations.

\[
	A u = f
\]
For small $A$, it is possible to use direct methods to solve these exactly (in exact arithmetic).
As is often the case in practice, the matrices from the finite element discretisation are too large to compute by direct methods.
Instead, many iterative methods have been developed to approximate the solution instead.
A particular class of methods, Krylov subspace methods, form the solution by building iterates from elements in a space composed of successive powers of the matrix $A$.
The Krylov space of dimension $k$ is 
\[
	\mathcal{K}_k(A, r_0) = \mathrm{span}(r_0, Ar_0, A^2 r_0, \ldots, A^{k-1} r_0).
\]
For example, the conjugate gradient method forms an orthogonal basis consisting of search directions.
Unfortunately, a restriction on the convergence of the CG algorithm is that the matrix $A$ must be symmetric and positive definite.
This is often not the case for a general problem.

Multigrid

The performance of multigrid methods is excellent for the related Poisson problem.
For small values of $k$ in \eqref{eqn:hh}, the matrix arising from the finite element discretisation is well-conditioned.
As $k$ becomes larger, however, the matrix becomes indefinite and multigrid fails to converge to a solution.


\subsection{Preconditioning}








\section{Project outline}

The multigrid solver for the Helmholtz equation exists for the equation in Cartesian coordinates in \texttt{oomph-lib}.
We would like to explore the performance of the solver for the Fourier decomposition of the Helmholtz equation in cylindrical coordinates.
Following this, we will also examine the performance of the solver using PMLs to handle reflections on the discretised boundary.


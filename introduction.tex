\chapter{Introduction}

% Applications and who cares

The fundamental nature of waves is the reason for their appearance in many different contexts in science, engineering, and industry.
For example, in electromagnetism, Maxwell's equations describe the physical properties of waves generated from electric and magnetic fields.
In acoustics, the propagation of sound waves may be modelled by similar equations.
Modelling of seismic waves can aid in the search for materials or objects underground and help to better understand earthquakes.
Understanding and modelling the scattering of waves is of great importance and practical use.
The equation underlying the behaviour of waves in all of these applications is the wave equation, given by
\begin{align}
	\nabla^2 u = \frac{1}{c^2}\frac{\partial^2 u}{\partial t^2}, \label{eqn:first}
\end{align}
where $c$ is the speed of the wave and the Laplacian is the sum of second partial derivatives,
\begin{align}
	\nabla^2 = \sum_{i=1}^3 \frac{\partial^2}{\partial x_i^2}.
\end{align}
This is a time-dependent, second order, hyperbolic partial differential equation.

The governing equation and the problem of interest in this project is the Helmholtz equation,
\begin{align}
	\nabla^2 u + k^2 u = 0,
\end{align}
where the parameter $k$ is the wavenumber and defined by $k=\omega/c$, where $\omega$ is the angular frequency of the wave.
The Helmholtz equation is a second order, elliptic partial differential equation that arises when seeking separable solutions of the wave equation.
Notably, this equation is now independent of time.
The physical interpretation of the solution $u$ depends on the particular application, but in general represents the amplitude of a wave with wavenumber $k$.
For our application, $u$ represents the acoustic pressure at a point in the domain.

Determining properties of an object located in a fluid is a problem applicable to many industries and fields.
Focusing on a single example, locating objects in large bodies of water such as sea mines or submarines require models for scattering of sonar waves.
To this end, consider a cylindrical object located within an infinite fluid.
We can represent the whole domain using cylindrical coordinates $(r, \phi, z)$, where $0\leq r < \infty$, $0 \leq \phi < 2\pi$, and $-\infty < z < \infty$.
If the object is located at the origin and aligned with the $z$-axis, then we can model the object in the domain with a vertically aligned cylindrical hole at the origin.
Assuming symmetry of the domain in the azimuthal direction, we can reduce the dimension of the problem.
Decomposing the solution into Fourier components in the azimuthal direction, we transform the Helmholtz equation in cylindrical coordinates into the Fourier decomposed Helmholtz equation.,
\begin{align}
	\nabla^2 u + k^2 u - \frac{N^2}{r^2}u = 0.
\end{align}


% Boundary conditions
We must also specify boundary conditions for the surface of the object, as well as any other boundaries in the domain.
For an elliptic problem, all boundary edges must be specified with either Dirichlet, Neumann, or Robin conditions.
A Dirichlet condition, where the known value of the solution is given on the boundary, $u|_{\partial\Omega}=u_0$.
This is sometimes known as a soft boundary, and can be interpreted as the boundary of an elastic wall that is displaced in response to a force on the outside.
A value of zero on the boundary is known as a pressure release boundary.
If the value here is nonzero, the boundary experiences some force from the other side.
For a Neumann boundary, the normal derivative $\vec{n}\cdot \nabla u |_{\partial \Omega} = \partial u / \partial \vec{n} |_{\partial \Omega}=g$ is given on the boundary.
The physical interpretation of this is a hard boundary that does not move in response to a force.
Lastly a Robin condition, giving both the value of the solution and the normal derivative on the boundary can be interpreted as a combination of the two above.



% Sommerfeld + truncation
In addition to boundary conditions, another condition on the solution must be imposed to ensure that the solution is unique.
This is due to the unboundedness of the domain.
The Sommerfeld radiation condition in three dimensions is given by
\begin{align}
	\lim_{|x|\rightarrow \infty} |x| \left( \frac{\partial}{\partial |x|} - ik \right) u(x) = 0. \label{eqn:sommerfeld}
\end{align}
This ensures that sources scatter to infinity and do not come from infinity \cite{sommerfeld}.

Because the problem is defined on an infinite domain, we must perform a truncation of the domain to find the solution numerically.
Choosing some appropriately sized region of interest surrounding the object, we restrict the domain to this region.
However, this causes problems with waves interacting with the boundary of the truncated domain.
Spurious waves from reflections on the outer walls propagate through the domain and the resulting solution is no longer correct.
It is possible to deal with these reflections by introducing an artificial absorbing layer to the boundary of the domain.
Several such layers exist, for instance absorbing boundary conditions (ABCs); Dirichlet-to-Neumann conditions (DtNs); and perfectly matched layers (PMLs).
These all attempt to dampen waves in the artificial layer surrounding the domain so that there are no reflections back into the domain.
The implementation of these absorbing layers is tricky due to their subtlety, and so will only be included given sufficient time.


% ----------------------------------------------
% How do we solve the problem


% FEM Discretisation
In order to solve the problem, we will use the C++ library \oomph  to find the solution using the finite element method \cite{oomph}.
This process involves creating a mesh of elements, then the discretisation of the equations on each element of the mesh.
The contribution of each element to the total problem is then combined and the result is a $n\times n$ linear system of equations,
\begin{align}
	A x = b.
\end{align}
This system of equations requires a solver in order to compute the solution.
For small $A$, it is possible to use direct methods to find $x$.
In \oomph, the default linear solver is the direct solver SuperLU.
The benefit of a direct solver is that the computed solution to the above equation is exact, modulo any floating point errors.
However, the computational cost becomes prohibitively expensive as the size of the problem increases, and even for modestly sized problems this solver fails.
The alternative approach is to use iterative methods to find an approximation to the solution.
A wide range of iterative methods have been developed to take advantage of the underlying structure of the matrix $A$.
One particular class of methods, Krylov subspace methods, form the solution by building iterates from elements in a space of successive powers of the matrix $A$ \cite{leveque}.
The Krylov space of dimension $p$ is 
\begin{align}
	\mathcal{K}_p(A, r_0) = \mathrm{span}(r_0, Ar_0, A^2 r_0, \ldots, A^{p-1} r_0).
\end{align}
Iterates from Krylov methods are linear combinations of the basis elements of the Krylov subspace.
This class of algorithms includes the conjugate gradient and GMRES methods, among several others.


% Preconditioning
Unfortunately, iterative methods suffer from a lack of robustness.
As such the effectiveness of such methods can be poor in general.
This issue can be addressed by using an approximation $P$ to $A$ that is relatively cheap to both find and invert.
By multiplying the original system by this approximation, we can readily improve the performance of iterative methods.
This approach is known as preconditioning, and the approximation $P$ is known as the preconditioner.

Applying a preconditioning matrix $P$ on the left, we have
\begin{align}
	P^{-1} A x = P^{-1} b. \label{eqn:precond_left}
\end{align}
Alternatively, a preconditioning matrix can be applied on the right.
In this case, we have
\begin{align}
	A P^{-1} P x = b. \label{eqn:precond_right}
\end{align}
This can be solved in two steps, 
\begin{align}
	P x = y, \qquad A P^{-1} y = b.
\end{align}

If we take $P=A$ as the preconditioner, we can solve the above system exactly for the case since $P^{-1}A=I$. 
This is optimal as it solves the original problem, however it is costly to obtain.
On the other hand, taking $P=I$ is cheap to obtain, but does not reduce the difficulty of solving the problem.
Instead, we would like an alternative such that $P$ is cheap to find and the preconditioned system is easier to solve than the original problem.

Many general preconditioning techniques have been developed that work for a variety of problems.
For example, preconditioning matrices based on the splitting of the matrix $A$ are simple to use and cheap to compute.
More specific preconditioning techniques have been developed in the case of specific problems.
These tend to be more complicated to derive and implement, but often will achieve better performance in the iterative method.
As a first attempt, it is a good idea to use a simple method to see the impact on the convergence times.
In the case where optimisation is crucial to the problem or a general preconditioner does not work, a specific preconditioner should be developed.


% Multigrid
An optimal class of iterative methods for solving problems arising from the discretisation of elliptic PDEs are the multigrid algorithms.
Wesseling describes the historical development of multigrid methods from their inception in 1967 up to 1985 \cite{wesseling}.
It is stated that the first mention of multigrid was by the Russian mathematician Fedorenko.
In his paper, Fedorenko developed a method for solving elliptic type partial differential equations, which he called the ``the method of alternating direction'' \cite{fedorenko}.
Later, Hackbusch independently developed multigrid methods and presented the rigorous mathematical framework for multigrid \cite{hackbusch}.

The performance of multigrid methods is excellent for the archetypal elliptic PDE, the Poisson problem,
\begin{equation}
	\nabla^2 u = f. \label{eqn:poisson}
\end{equation}
Indeed, it is an optimal solver with time-complexity $O(n)$, scaling linearly in the number of degrees of freedom of the problem.
The Helmholtz equation is exactly the Poisson equation when $k^2=0$ and $f=0$.
The matrix in the finite element discretisation of the Helmholtz equation becomes indefinite and ill-conditioned as $k^2$ becomes large.
This causes issues with linear solvers, in particular it causes multigrid to fail to converge to a solution.
To handle the ill-conditioned system, we use a preconditioner so that the problem may be solved by multigrid.


% Multigrid as a preconditioner, CSLP
Work on overcoming the issues in using multigrid to solve the Helmholtz equation for large values of $k^2$ has been successful.
The method of solving the problem is to precondition a Krylov subspace method using a shift into the complex plane.
Using a transformation of the Helmholtz equation to
\begin{align}
	\nabla^2 u + (1+i\alpha)k^2 u = 0,
\end{align}
where $\alpha$ controls the complex shift.
This is known as the complex-shifted Laplacian preconditioner (CSLP).
Multigrid performs well in solving this problem, as shown in the convergence results obtained by Erlangga, Oosterlee, and Vuik \cite{cslp1}.
Among the class of preconditioners discussed, the CSLP is the most effective in preconditioning the Helmholtz problem for large wavenumbers.
In another paper by the same authors, they used a multigrid to compute the preconditioner to solve the system using an outer iteration of FGMRES \cite{cslp2}.
This was done since at each iteration of the outer iterative method the preconditioner will change, so the method must be `flexible'.
Saad created the Flexible GMRES iterative method, FGMRES \cite{fgmres}.
The multigrid solver for the Helmholtz equation exists for the equation in Cartesian coordinates in \oomph.
The aim for this project is to implement this method for the Fourier decomposed version of the Helmholtz equation in axisymmetric cylindrical coordinates and compare its performance to related problems.





% -----------------------------------------------

\section{Project outline}

Chapter one contains the introduction to the project.
This has hopefully given you, the reader, the modern and historical context of the problem of interest, as well as the aims and objectives to motivate the rest of this dissertation.

The focus of chapter two is to derive the mathematics of the problem.
Here we will give the necessary mathematical overview and description of the problem and the derivation of equations.
Also here will be issues associated with the practicalities of finding a solution.

Chapter three aims to cover the necessary high level facets of the finite element method for the the problem in question.
Also covered are relevant references to the FEM implementation within \oomph and how the mathematical description of the FEM is transferred to an object-oriented code able to solve PDEs numerically.

Chapter four will motivate multigrid and outline the mathematics of the multigrid algorithm for solving the Helmholtz equation.
In particular, we discuss the application of multigrid in finding the solution to the finite element discretisation of the Helmholtz equation.

The project will come together at the end with a discussion of the results.
Here, a summary of findings and an overview of the work completed in the project will be given, as well as comparison to related problems and the current literature.

Finally, we will complete the dissertation with concluding remarks and outline steps for moving forward from the end of the project.
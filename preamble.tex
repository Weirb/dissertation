\title{Finite-element-based solution of the Fourier-decomposed Helmholtz equation using a multigrid preconditioner}
\author{William Weir}
\school{Mathematics}
\faculty{Engineering and Physical Sciences}
\def\wordcount{xxxxx}

% Uncomment the line below to suppress the `List of Tables' page (optional)
% \tablespagefalse

% Uncomment the line below to suppress the `List of Figures' page (optional)
%\figurespagefalse

% Uncomment the line below to use a customised Declaration statement
%\def\declaration{All the work in this thesis has been sourced from Google.}

\beforeabstract

% Abstract

The Helmholtz equation arises in many contexts of several industries.
Since waves feature in electromagnetism, acoustics, and many other fields, any general work has wide application.
Therefore there is much need for a fast and efficient solver for this problem.

Multigrid is an optimal solver applied to the discretisation of elliptic PDEs.
This class of equations includes the Poisson equation.
However, multigrid faces issues with the related Helmholtz equation in cylindrical coordinates.
For example, large values of the wavenumber cause the discretisation matrix to become indefinite.
This has largely been overcome by introducing a shift into the complex plane and using a multigrid preconditioner for an outer Krylov method.

This project extends existing work on the Cartesian multigrid solver for the Helmholtz equation.
We achieve this by the implementation of the solver for the Fourier decomposed Helmholtz equation in cylindrical coordinates.
We present convergence results to aid in a comparison between our solver and the Cartesian solver for several related problems.

\afterabstract

\prefacesection{Acknowledgements}

I would like to thank my supervisor, Matthias Heil.
The help and support that I have received has been invaluable during this project.

Also his PhD students, Puneet Matharu and Jonathan Deakin.
Thank you for taking the time to talk through the many problems that I came up against.

Finally, I am grateful for all of the interesting discussions and biscuits at the Monday afternoon \texttt{oomph-lib} lunches.
These have introduced me to a wide range of topics that I would otherwise have not had access to.
